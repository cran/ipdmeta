Mixed-level meta-analysis (MLMA) describes a quantitative summary of
evidence at disparate levels with some trials providing
individual-level data and other trials summary data. The \texttt{mlma}
function performs MLMA estimation when the outcome is a time-to-event
and aggregate data is in the form of a set of survival estimates
within treatment group, within study. 

The individual patient data (IPD) model is a proportional hazards
mixed model (PHMM) allowing a general random effects structure with multivariate normal
frailties. The study-level model is a multivariate mixed model on the
complementary-log-log of the study-specific survival estimates. This
study-level model is the implied linear relation based on the
assumption that all outcomes follow the PHMM at the patient level. 

Through
combined likelihood maximization, both evidence levels contribute to
the estimation of shared fixed effects and the frailty variance
structures. The estimation uses an MCEM approach with importance
sampling following the same procedure are for the PHMM analysis
implemented by \texttt{coxmcem}. Separation of the patient-level and study-level effects is
also possible if there is concern of non-equivalence in the risk
associations at the aggregate level versus the subject level.

\subsection{Usage}

Consider a mixed-level dataset consisting of 8 IPD and 2
study-level RCTs. The model for the PHMM has treatment main effect. A bivariate frailty
for baseline and treatment effect by trial is the random component.

See \texttt{example(ipd.data)} and \texttt{example(meta.data)} for a
quick visual display of the nature of the mixed-level data for this
example.

To obtain the MLMA estimates, I specify model formula for the study- and
individual-level models. The individual-level model is a hazard-based
model and takes a formula like that for \texttt{survfit} or \texttt{coxph}.
The study-level model is based on aggregated survival proportions with
their squared standard errors (\texttt{sigma2}). The random component
uses indicated the frailty structure using the same form as for
\texttt{coxme} or \texttt{lme}. The argument
\texttt{study.group.interaction} is the factor that is the
cluster, here \texttt{group}, and the treatment group indicator. This
factor identifies the membership to the within treatment, within study
groups for which separate survival estimates have been collected. 

To gain some guidance in selection of the fixed effects model I can
make use of a less computationally intense analysis with the patient-level data. 

\begin{Schunk}
\begin{Sinput}
> fit.coxph <- coxph(Surv(time,event)~trt*x,ipd.data)
> fit.coxph
Call:
coxph(formula = Surv(time, event) ~ trt * x, data = ipd.data)


          coef exp(coef) se(coef)      z       p
trt   -0.45666     0.633   0.0583 -7.832 4.8e-15
x      0.02494     1.025   0.0291  0.856 3.9e-01
trt:x -0.00906     0.991   0.0419 -0.216 8.3e-01

Likelihood ratio test=70.3  on 3 df, p=3.66e-15  n= 1600 
\end{Sinput}
\end{Schunk}

This suggests that the candidate covariate is not important. I can
verify this with a trial run of \texttt{mlma}. The model components are
as follows.

The patient-level PHMM model is

\[
\lambda_i(t|b) = \lambda_0(t) \mbox{exp}\lbrace x_{trt,i} \beta_1 +
x_i \beta_2 + b_{k(i),1} + b_{k(i),2} x_{trt,i} \rbrace
\]

\noindent where k(i) is the study membership for the ith subject. Each
$b_k$ is a bivariate normal frailty with general variance structure.

The study-level model for each KM survival estimate provided by study
is

\[
g(s_i|\tilde{b}) = \mbox{log}(t_i) + \tilde{x}_{trt,i} \beta_1 +
\tilde{x}_i \beta_2 + \tilde{b}_{j(i)}+\tilde{b}_{j(i),2} \tilde{x}_{trt,i} +\epsilon_i
\]

\noindent Here $g(x) = log(-log(x))$  and $\epsilon_i \sim
N(0,\sigma_i^2)$ where $\sigma_i^2$ is considered known. The residual
variance is determined by the standard error for the KM estimate and
applying the delta method for the complementary-log-log
transform. When there are multiple estimates from the same study, the
variance structure accounts for correlation within treatment group,
within trial. The methodology is in keeping with \cite{Arends2008a}.

The covariates for the study-level model are grouped which is why I
have used the notation $\tilde{x}$ to distinguish them from the
patient-level model. Thus, the patient level factor $x$ when
aggregated into the cluster sample means is denoted as $\tilde{x}$.

Implementation of the model proceeds as follows:

\begin{Schunk}
\begin{Sinput}
set.seed(123321)
data(ipd.data)
data(meta.data)

fit <- mlma(
    Surv(time,event)~trt+x,
    surv~-1+log(time)+trt+x,
    random=~(1+trt|group),
    ipd.groups=8,
    meta.groups=2,
    ipd.data=ipd.data,
    meta.data=meta.data,
    sigma2=meta.data$sigma2,
    study.group=meta.data$sub.group,
    max.iter=10,
    est.delta=.01,
    min=300
)

> #WALD TEST FOR FIXED EFFECTS
> fit$coef
        trt           x   log(time) 
-0.63023004  0.00326478  1.04518270 

> fit$coef/sqrt(diag(fit$var$coef))
        trt           x   log(time) 
-6.45590592  0.09845826 10.47860368 

> sqrt(fit$vcov)
          [,1]      [,2]
[1,] 0.6430685 0.3374159
[2,] 0.3374159 0.3759782
> 

> #WALD TEST FOR FRAILTY VARIANCES
> sqrt(diag(fit$vcov))
[1] 0.6430685 0.3759782
> diag(fit$vcov)/sqrt(fit$var$vcov)
         [,1]
[1,] 2.062913
[2,] 1.849326
\end{Sinput}
\end{Schunk}

There is no evidence of a significant effect for the covariate
\texttt{x} so I remove this factor and re-fit with just the treatment
effect. I examine the estimates with the revised fit.

\begin{Schunk}
\begin{Sinput}
> fit$coef
       trt  log(time) 
-0.5352532  1.0558993 

> fit$coef/sqrt(diag(fit$var$coef))
      trt log(time) 
-6.125604 10.800622 
> sqrt(diag(fit$vcov))
[1] 0.6342262 0.3850056
> diag(fit$vcov)/sqrt(fit$var$vcov)
         [,1]
[1,] 2.067657
[2,] 1.987731
\end{Sinput}
\end{Schunk}

All of the model parameters contribute important information to
survival outcomes. 

\begin{Schunk}
\begin{Sinput}
> fit$est.con[(fit$iter-5):fit$iter]
[1] 0.09792774 0.09908291 0.06861460 0.07117281 0.05336713 0.12839772
\end{Sinput}
\end{Schunk}

The convergence criterion was not met, however, so I will want to
adjust the number of MC samples of the number of iterations before
drawing firm conclusions from \texttt{fit}.

When the converged estimates have been obtained I can compare the
baseline hazard implied by each model, which is another check of the
consistency between the patient-level and study-level data (Figure 1).

\begin{Schunk}
\begin{Sinput}
H <- bas.haz(
             Surv(time,event)~trt,
             ~-1+log(t),
             ipd.data,
             fit$coef,
             fit$var$coef,
             )

#PATIENT-LEVEL BASELINE HAZARD

plot(H$ipd.survfit,fun="cumhaz",ylab="H(t)",bty="n")

#STUDY-LEVEL BASELINE HAZARD WITH 95\% CI

lines(x=H$meta.bas.haz$time,y=H$meta.bas.haz$lower,type="l",lty=2)
lines(x=H$meta.bas.haz$time,y=H$meta.bas.haz$est,type="l",col="blue")
lines(x=H$meta.bas.haz$time,y=H$meta.bas.haz$upper,type="l",lty=2)

legend("topleft",legend=c("Study-level","Patient-level"),lty=1,col=c("blue","black"))
\end{Sinput}
\end{Schunk}

If a fixed effects model was wanted, I can obtain it through the
\texttt{fixed} argument.

\begin{Schunk}
\begin{Sinput}
fit.fixed <- mlma(
    Surv(time,event)~trt,
    surv~-1+log(time)+trt,
    random=~(1+trt|group),
    ipd.groups=8,
    meta.groups=2,
    ipd.data=ipd.data,
    meta.data=meta.data,
    sigma2=meta.data$sigma2,
    study.group=meta.data$sub.group,
    fixed=TRUE
)

> fit.fixed$coef
      trt log(time) 
-0.476680  1.127154 
\end{Sinput}
\end{Schunk}

This provides a means of constructing a likelihood ratio test for the
frailty variances, compared the fixed effects model likelihood to the
mixed effects model. 

\begin{Schunk}
\begin{Sinput}
> #LIKELIHOOD RATIO TEST STATISTIC
> -2*(fit.fixed$loglik-fit$loglik[fit$iter])
         [,1]
[1,] 659.4629
\end{Sinput}
\end{Schunk}

Comparing this statistic to a $\chi^2(3)$ for the 3 variance parameters of the
general covariance-variance structure for the frailties of the mixed
model gives some guidance on the presence of significant intercluster
variation. In this case, there is strong evidence of heterogeneity
among outcomes.

\subsection{Ecological Bias}

If I were concerned that my conclusions about the covariate \texttt{x}
were mislead by the presence of a study-level bias, I can separate
the parameterization for each evidence-level. I do this by
introducing a centered covariate into the individual level model. Note
that the model formulae have to have the same term label for all
shared effects.

The patient-level PHMM model changes to

\[
\lambda_i(t|b) = \lambda_0(t) \mbox{exp}\lbrace x_{trt,i} \beta_1 +
(x_i - x_i^{*}) \beta_2 + x_i^{*} \beta_3 + b_{k(i),1} + b_{k(i),2} x_{trt,i} \rbrace
\]

\noindent where $x_i^{*}$ is the k(i) sample average of the covariate,
the grouped effect. Thus, $\beta_2$ is the patient-level effect while
$\beta_3$ is the study-level effect. Accordingly, the study-level
model is

\[
g(s_i|\tilde{b}) = \mbox{log}(t_i) + \tilde{x}_{trt,i} \beta_1 +
\tilde{x}_i^{*} \beta_3 + \tilde{b}_{j(i),1}+\tilde{b}_{j(i),2} \tilde{x}_{trt,i} +\epsilon_i
\]

\noindent where the notation of the covariate has changed to
correspond with the patient-level model, but the elements are still
the study-level sample averages for the covariate \texttt{x}.

\begin{Schunk}
\begin{Sinput}
n <- table(ipd.data$group)
ipd.data$x.star <- rep(tapply(ipd.data$x,ipd.data$group,mean),n)
names(meta.data)[which(names(meta.data)=="x")] <- "x.star"

fit.bias <- mlma(
    Surv(time,event)~trt+I(x-x.star)+x.star,
    surv~-1+log(time)+trt+x.star,
    random=~(1+trt|group),
    ipd.groups=8,
    meta.groups=2,
    ipd.data=ipd.data,
    meta.data=meta.data,
    sigma2=meta.data$sigma2,
    study.group=meta.data$sub.group,
    max.iter=20,mc=1.3,
    est.delta=.01,
    df=25,
    min=500
)

> fit.bias$coef
          trt I(x - x.star)        x.star     log(time) 
 -0.553248425   0.006493564   0.306708999   1.073181569 

> ci(c(0,-1,1,0),fit.bias$coef,fit.bias$var$coef,f=function(x){x})
      low point.est      high 
0.2795519 0.3002154 0.3208790 
\end{Sinput}
\end{Schunk}

I obtain the 95\% CI for the contrast between the study- and
patient-level effect for the covariate \texttt{x.star}. There is evidence
that a positive relation at the study-level is present, while no
individual-level relation is demonstrated.
